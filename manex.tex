\documentclass{beamer}
\usetheme{Manchester}

\newtheorem{remark}[theorem]{Remark}
\newtheorem{caveat}[theorem]{Caveat}
\newtheorem{rmcorollary}[theorem]{Corollary}

\begin{document}

\title{A Beamer Theme For Lovers of The Colour Purple}
\author[M.Kambites]{Mark Kambites}
\institute[Manchester]{School of Mathematics, \ University of Manchester}
\date[INPNBD'08]{The 35th Annual Conference of the \\
International Network of People with Nothing Better To Do}

\frame{
  \titlepage
}

\begin{frame}
   \frametitle{The Manchester Beamer Theme}

\uncover<1->{\begin{definition}
\textit{Manchester} is a simple beamer theme for creating presentations
using the University of Manchester colours and logo.
\end{definition}}

\uncover<2->{\begin{remark}
It is based on the Boadilla theme, designed by Manual Carro, which is
distributed with Beamer.
\end{remark}}

\uncover<3->{
\begin{remark}
As you can see, the default style is quite minimal, showing only
the speaker/author's name, title, Manchester logo and slide number.
\end{remark}}

\uncover<4->{\begin{remark}
As with Boadilla, the option \textrm{\texttt{secheader}} adds a bar at the top
showing the section and subsection titles.
\end{remark}}
\end{frame}


\begin{frame}
\frametitle{The Downside}

\uncover<1->{
\begin{caveat}
The theme is still very much experimental, and used at your own risk. I
take no responsibility if it makes you look foolish in front of all the
leading experts in your field. \textrm{:-)}
\end{caveat}
}

\uncover<2->{
\begin{rmcorollary}
Feedback is welcome!
\end{rmcorollary}
}

\uncover<3->{
\begin{caveat}
The Manchester logo may look blocky if projected very large.
I am trying to obtain a higher resolution version.
\end{caveat}
}

\end{frame}




\begin{frame}
   \frametitle{Installation and Usage}
\begin{enumerate}
\item<1-> Download the files
\begin{itemize}
\item \texttt{beamerthemeManchester.sty}
\item \texttt{manchesterlogo.jpg}
\item \texttt{manex.tex}
\end{itemize}
from Mark's homepage, which you have presumably already found.

\item<2-> Save them somewhere in your \TeX \ path (or if you don't know
what that means, in the directory where you will put your
presentation).

\item<3-> Create your presentation (eg.~by modifying \texttt{manex.tex}).

\item<4-> Process it with \texttt{pdflatex}.

\item<5-> If it doesn't look right, go to (3).

\item<6-> Knock 'em dead with your enthralling presentation.

\item<7-> Send feedback to \texttt{Mark.Kambites@manchester.ac.uk}.
\end{enumerate}
\end{frame}

\end{document}
